\chapter*{\abstractname}
\addstarredchapter{\abstractname} % minitoc
\makecseabstract


\noindent Η ολοένα και αυξανόμενη ανάγκη για μεγαλύτερη επεξεργαστική ισχύ οδήγησε στη δημιουργία των συστημάτων \textit{μη ομοιόμορφης προσπέλασης μνήμης} (NUMA) τα οποία αποτελούν αρχιτεκτονική εξέλιξη των κοινών \textit{συμμετρικών πολυεπεξεργαστών} (SMPs) και τα οποία είναι εφοδιασμένα με μεγάλους αριθμούς επεξεργαστικών πυρήνων. Τα συστήματα αυτά παρέχουν κοινόχρηστο χώρο διευθύνσεων και συνεπώς επιτρέπουν την ανάπτυξη προγραμμάτων με διαδεδομένες \textit{διεπαφές προγραμματισμού εφαρμογών} (APIs) όπως το OpenMP. Λόγω της πολύπλοκης αρχιτεκτονικής οργάνωσης των συστημάτων NUMA, η επίτευξη των βέλτιστων δυνατών επιδόσεων συνήθως απαιτεί την εκμετάλλευση πληροφοριών που σχετίζονται με την τοπολογία του υποκείμενου συστήματος, δηλαδή με το πώς είναι οργανωμένο το υλικό. Ήδη από την έκδοση 4.0 του OpenMP, άρχισαν να προδιαγράφονται λειτουργίες όπως τα OpenMP places και OpenMP processor binding policies οι οποίες σχετίζονται με την τοπολογία και επιτρέπουν στο χρήστη να ελέγξει τον τρόπο ανάθεσης των νημάτων στα διαθέσιμα επεξεργαστικά στοιχεία. Στα πλαίσια της παρούσας διπλωματικής εργασίας, θα ασχοληθούμε με την υλοποίηση των λειτουργιών OpenMP places και OpenMP processor binding policies, καθώς επίσης θα προσπαθήσουμε να βελτιώσουμε τις υπάρχουσες λειτουργίες συγχρονισμού, όπως οι \textit{κλήσεις φραγής} (barriers), ώστε να λειτουργούν αποδοτικά σε συστήματα NUMA.


\bigskip