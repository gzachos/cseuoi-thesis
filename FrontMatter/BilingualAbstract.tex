\chapter*{\csebilabstract}
\addstarredchapter{\csebilabstract} % minitoc
\makecsebilabstract

\noindent The growing need for more processing power has led to the creation of \textit{non-uniform memory access} (NUMA) systems which constitute the architectural evolution of \textit{symmetric multiprocessors} (SMPs) and are equipped with a large number of processing cores. These systems provide a shared-address space and therefore can be programmed %allow the development of programs
using widespread \textit{application programming interfaces} (APIs) such as the OpenMP API. Due to the complex architectural organization of NUMA systems, achieving the best possible performance usually requires using information related to the topology of the underlying system, that is, information about how its hardware is organized. Since version 4.0 of the OpenMP specification, topology-related features such as \textit{OpenMP places} and \textit{OpenMP processor binding policies} have beed introduced in order to allow users to control the assignment of threads to the available processing elements. In the context of this disseration, the OpenMP places and OpenMP processor binding policies features were fully implemented. In addition, synchronization functions, specifically barriers, were redesigned to function efficiently on NUMA systems.

\bigskip
