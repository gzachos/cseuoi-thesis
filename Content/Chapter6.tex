\chapter{Σύνοψη και Μελλοντική Εργασία}

\section{Ανακεφαλαίωση}
Η ανάγκη για όλο και μεγαλύτερη επεξεργαστική ισχύ οδήγησε στη δημιουργία των συστημάτων μη ομοιόμορφης προσπέλασης μνήμης (NUMA), που αποτελούν αρχιτεκτονική εξέλιξη των συμμετρικών πολυεπεξεργαστών (SMPs). Τα συστήματα αυτά παρέχουν κοινόχρηστο χώρο διευθύνσεων και συνεπώς επιτρέπουν τη χρήση διαδεδομένων μοντέλων προγραμματισμού όπως το OpenMP. Λόγω της κατανεμημένης από φυσική άποψη οργάνωση της μνήμης των συστημάτων αυτών, οδηγούμαστε στην εκμετάλλευση πληροφοριών που σχετίζονται με την τοπολογία του υποκείμενου συστήματος για την επίτευξη των βέλτιστων δυνατών επιδόσεων.

Κάτι τέτοιο έρχεται σε αντίθεση με την υψηλού επιπέδου διεπαφή προγραμματισμού εφαρμογών OpenMP, στο πλαίσιο της οποίας χρησιμοποιούνται αφαιρέσεις (abstractions) ώστε να είναι δυνατή η συγγραφή φορητών προγραμμάτων χρήστη. Παρόλα αυτά, λόγω της ανάγκης χρήσης πληροφοριών που σχετίζονται με την τοπολογία του συστήματος, από την έκδοση 4.0 του προτύπου OpenMP και έπειτα, άρχισαν να προδιαγράφονται λειτουργίες που σχετίζονται με την τοπολογία του υποκείμενου συστήματος. Οι λειτουργίες αυτές επιτρέπουν τον ορισμό συνόλων επεξεργαστών (OpenMP places) στα οποία μπορούν να ανατεθούν τα νήματα OpenMP βάσει διάφορων πολιτικών, γνωστών ως OpenMP processor binding policies.

Στα πλαίσια της παρούσας διπλωματικής εργασίας, υλοποιήθηκαν πλήρως τα OpenMP places και OpenMP processor binding policies όπως προδιαγράφονται από την πιο πρόσφατη έκδοση του προτύπου OpenMP και συγκεκριμένα την έκδοση 5.1 (Νοέμβριος 2020). Η ικανότητα ελέγχου του τρόπου ανάθεσης των νημάτων σε σύνολα από επεξεργαστές, επιτρέπει τον διαμοιρασμό των νημάτων στους διαθέσιμους επεξεργαστές ανάλογα με τη λογική οργάνωση της εργασίας που πρέπει να επιτελέσει το πρόγραμμα χρήστη. Επιπλέον, υλοποιήθηκε ένας νέος αλγόριθμος barrier για τον ερευνητικό μεταφραστή OMPi που υποστηρίζει το OpenMP. Σκοπός του νέου αλγορίθμου ήταν η δυνατότητα συγχρονισμού των νημάτων σε συστήματα NUMA με κλιμακώσιμο τρόπο, καθώς το πλήθος των κόμβων αυξάνεται.

Τα μετροπρογράμματα ελέγχου της ορθότητας της υλοποίησης αλλά και μέτρησης του κόστους σε χρόνο (overhead) που απαιτείται για το συγχρονιμό με χρήση barrier, έδειξαν ότι ο νέος barrier έχει τη δυνατότητα να χρησιμοποιείται αποδοτικά για το συγχρονισμό νημάτων που εκτελούνται σε πολλαπλούς κόμβους NUMA. Ιδιαίτερα σημαντικό είναι το γεγονός ότι ο νέος αλγόριθμος barrier σε σχέση με τον κλασικό όχι μόνο είναι λιγότερο κοστοβόρος από άποψη χρόνου, αλλά έχει και καλές ιδιότητες κλιμάκωσης, καθώς για κάθε επιπλέον κόμβο που χρησιμοποιείται, το επιπλέον overhead αυξάνεται με μικρό ρυθμό.

\section{Μελλοντική Εργασία}
