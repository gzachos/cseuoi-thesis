\chapter{Πειραματική Αξιολόγηση}
\label{ch:Experimental Evaluation}
Οι διάφορες υλοποιήσεις που πραγματοποιήθηκαν στα πλαίσια της τρέχουσας διπλωματικής εργασίας αξιολογήθηκαν πειραματικά με τη χρήση μετροπρογραμμάτων (benchmarks) τόσο για την επιβεβαίωση της ορθότητας της υλοποίησης, όσο και για τη μέτρηση των επιδόσεων που επιτεύχθηκαν.


\section{Περιγραφή Συστημάτων}
\label{sec:Systems Description}
Η εκτέλεση των πειραμάτων έγινε σε υπολογιστικά συστήματα που εκπροσωπούν διαδεδομένες και ενδιαφέρουσες αρχιτεκτονικές οργανώσεις, όπως για παράδειγμα συστήματα τύπου SMP και NUMA. Τα χαρακτηριστικά των συστημάτων αυτών, τόσο από άποψη υλικού, όσο και από άποψη λογισμικού, φαίνονται στους Πίνακες \ref{tab:exp-systems-hardware} και \ref{tab:exp-systems-software}. Η αρχιτεκτονική των επεξεργαστών όλων των συστημάτων είναι η \textit{x86-64} ενώ το μέγεθος μιας γραμμής της κρυφής μνήμης είναι 64 bytes.

Οι μεταφραστές οι οποίοι χρησιμοποιήθηκαν για τη συγκριτική αξιολόγηση του μεταφραστή OMPi είναι ο \textit{GNU C Compiler} (GCC), ο \textit{Intel C Compiler} (ICC)\footnote{Εγκαταστάθηκε μέσω των oneAPI Toolkits της Intel\textsuperscript{\textregistered}.} και ο \textit{Clang}. Ο πηγαίος κώδικας του OMPi μεταφράστηκε με τον GCC, ενώ τα απαιτούμενα πακέτα λογισμικού είναι διαθέσιμα στο Παράρτημα \ref{app:OMPi's software requirements}.

\begin{table}
	\centering
		\begin{tabular}{|c||c|c|c|c|c|c|}
		\hline
		Hostname & Proc. Mfr. & NUMA nodes & Sockets & Cores & H/W threads & RAM \\
		\hline \hline
		\texttt{parade} & Intel & 4 & 4 & 64 & 128 & 256 GB \\
		\hline
		\texttt{paragon} & AMD  & 4 & 2 & 24 & 24 & 16 GB \\
		\hline
		\texttt{opti3060} & Intel & 1 & 1 & 4 & 4 & 8 GB \\
		\hline
		\texttt{ideapad} & Intel & 1 & 1 & 2 & 4 & 8 GB \\
		\hline
		\end{tabular}
		\caption{Χαρακτηριστικά υλικού των πειραματικών συστημάτων.}
		\label{tab:exp-systems-hardware}
\end{table}

\begin{table}
	\centering
		\begin{tabular}{|c||c|c|c|c|c|c|c|}
		\hline
		Hostname & OS & GCC & ICC & Clang & hwloc & Linux kernel \\
		\hline \hline
		\texttt{parade} & CentOS 8 & \texttt{8.4.1} & \texttt{2021.3.0} & \texttt{11.0.0} & \texttt{2.2.0} & \texttt{4.18.0} \\
		\hline
		\texttt{paragon} & CentOS 8 & \texttt{8.4.1} & \texttt{2021.3.0} & \texttt{11.0.0} & \texttt{2.2.0} & \texttt{4.18.0} \\
		\hline
		\texttt{opti3060} & Ubuntu 18.04.3 & \texttt{7.5.0} & \texttt{N/A} & \texttt{N/A} & \texttt{1.11.9} & \texttt{5.0.0} \\
		\hline
		\texttt{ideapad} & Mint 20.2 & \texttt{9.3.0} & \texttt{2021.3.0} & \texttt{10.0.0} & \texttt{2.1.0} & \texttt{5.4.0} \\
		\hline
		\end{tabular}
		\caption{Χαρακτηριστικά λογισμικού των πειραματικών συστημάτων.}
		\label{tab:exp-systems-software}
\end{table}


\subsection{Parade}
Ο Parade είναι ένα σύστημα Dell PowerEdge R840 με τέσσερις κόμβους NUMA. Κάθε κόμβος διαθέτει 64 GBs μνήμης και έναν επεξεργαστή \textit{Intel\textsuperscript{\textregistered} Xeon\textsuperscript{\textregistered} Gold 6130} ο οποίος αποτελείται από 12 πυρήνες και ιεραρχία κρυφών μνημών τριών επιπέδων (L1i \& L1d, L2, L3). Τα επίπεδα ένα και δύο των κρυφών μνημών είναι κοινά ανά πυρήνα, ενώ το επίπεδο τρία είναι κοινό για όλους τους πυρήνες του επεξεργαστή. Επίσης, κάθε πυρήνας περιέχει δύο H/W threads. Η σχηματική αναπαράσταση της τοπολογίας του φαίνεται στο Σχήμα \ref{fig:parade-topo}.

\subsection{Paragon}
Ο Paragon είναι ένα σύστημα με δύο επεξεργαστές \textit{AMD Opteron\textsuperscript{\texttrademark} Processor 6166 HE}, καθένας από τους οποίους περιλαμβάνει δύο κόμβους NUMA. Κάθε κόμβος διαθέτει 6 πυρήνες του ενός H/W thread και ιεραρχία κρυφής μνήμης τριών επιπέδων αντίστοιχη με τους κόμβους του συστήματος Parade που είδαμε προηγουμένως. Η σχηματική αναπαράσταση της τοπολογίας του φαίνεται στο Σχήμα \ref{fig:paragon-topo}.

Ο λόγος που κάθε επεξεργαστής περιλαμβάνει δύο κόμβους είναι επειδή ουσιαστικά αποτελείται από δύο κυκλώματα επεξεργαστών (dies) των έξι πυρήνων το καθένα, τα οποία συνδέονται μεταξύ τους με ένα δίκτυο διασύνδεσης χαμηλής καθυστέρησης και υψηλού εύρους ζώνης που ονομάζεται HyperTransport, με σκοπό να δημιουργηθεί ένα ολοκληρωμένο κύκλωμα επεξεργαστή με 12 πυρήνες \cite{conway2010cache}. Κάθε die μπορεί να επικοινωνήσει απευθείας με τη μνήμη\footnote{Επειδή περιλαμβάνει ελεγκτή μνήμης (memory controller).}, οπότε λόγω της ύπαρξης δικτύου διασύνδεσης μεταξύ των dies, κάθε επεξεργαστής μπορεί να θεωρηθεί ως ένα σύστημα NUMA δύο κόμβων.


\begin{figure}[t]
	\centering
	\includegraphics[width=0.8\textwidth]{Figures/paragon-topo.pdf}
	\linebreak
	\caption{Η τοπολογία του συστήματος Paragon.}
	\label{fig:paragon-topo}
\end{figure}


\section{Τοπολογία}
\label{sec:Topology}


\section{Barrier}
\label{sec:Barrier}

