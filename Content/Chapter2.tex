\chapter{Η διεπαφή προγραμματισμού OpenMP}
\label{ch:OpenMP API}

\section{Εισαγωγή στο OpenMP}
\label{sec:Introduction to OpenMP}
Όπως αναφέρθηκε ήδη στην υποενότητα Χ, η διεπαφή προγραμματισμού εφαρμογών OpenMP (Open Multi-Processing) αναπτύχθηκε για τη διευκόλυνση της ανάπτυξης πολυνηματικών εφαρμογών για συστήματα κοινόχρηστης μνήμης. Οι γλώσσες οι οποίες υποστηρίζονται είναι οι C, C++ και Fortran.
To OpenMP αποτελείται από:
\begin{itemize}
	\item Oδηγίες (directives): Συνιστούν οδηγίες προς τον μεταφραστή για το πώς και τι να εκτελέσει πολυνηματικά. Στις γλώσσες C/C++ χρησιμοποιείται ο μηχανισμός που είναι γνωστός ως pragmas και απευθύνεται στον προεπεξεργαστή. Αυτές οι οδηγίες προστίθενται στο υπάρχον σειριακό πρόγραμμα και μπορούν να αγνοηθούν από έναν μεταφραστή που δεν τις υποστηρίζει. Αυτό είναι μεγάλο προσόν καθώς το ίδιο πρόγραμμα μπορεί να εκτελεστεί σειριακά ή παράλληλα.
	\item Ρουτίνες βιβλιοθήκης: σύνολο συναρτήσεων οι οποίες βοηθούν στη διαχείριση των χαρακτηριστικών των νημάτων και του περιβάλλοντος εκτέλεσης. Για παράδειγμα, η συνάρτηση \texttt{omp\_set\_num\_threads(int)} καθορίζει το πλήθος των νημάτων που θα συμμετάσχουν σε επερχόμενη πολυνηματική εκτέλεση (παράλληλη περιοχή).
	\item Μεταβλητές περιβάλλοντος: χρησιμοποιούνται για τον καθορισμό διάφορων χαρακτηριστικών των νημάτων και του περιβάλλοντος εκτέλεσης. Οι τιμές των μεταβλητών περιβάλλοντος οριστικοποιούνται στην αρχή της εκτέλεσης και χρησιμοποιούνται ως προκαθορισμένες τιμές. Κάποιες από αυτές τις προκαθορισμένες αυτές τιμές μπορούν να τροποποιηθούν σε χρόνο εκτέλεσης με χρήση των διαθέσιμων ρουτινών βιβλιοθήκης.
\end{itemize}

Από τη στιγμή που η παραλληλοποίηση ενός σειριακού προγράμματος μπορεί να γίνει με την απλή προσθήκη οδηγιών στο υπάρχοντα κώδικα, η διαδικασία της παραλληλοποίησης απλοποιείται σε μεγάλο βαθμό και μπορεί να γίνει σταδιακά (π.χ. παραλληλοποίηση ενός βρόγχου for τη φορά) και χωρίς τη χρήση διαφορετικής λογικής όπως για παράδειγμα θα γινόταν με χρήση των POSIX Threads. 

Λαμβάνοντας υπόψην ότι η διαχείριση των νημάτων μετατίθεται από τον προγραμματιστή στο μεταφραστή, καθώς επίσης ότι απλοποιούνται ουσιώδη ζητήματα ενός παράλληλου προγράμματος όπως η επίτευξη συγχρονισμού και αμοιβαίου αποκλεισμού, γίνεται εύκολα αντιληπτό ότι το OpenMP είναι ένα προσιτό εργαλείο ακόμα και για άτομα χωρίς μεγάλη εμπειρία στον παράλληλο προγραμματισμό. Αυτό το χαρακτηριστικό του OpenMP το κάνει ιδιαίτερα διαδεδομένο σε χρήστες που το υπόβαθρό τους διαφέρει από αυτό της επιστήμης της πληροφορικής, όπως για παράδειγμα φυσικοί, χημικοί, αστρονόμοι κόκ. Ταυτόχρονα όμως, η απόκρυψη των λεπτομερειών χαμηλού επιπέδου είναι πιθανό να καταστήσει σε ορισμένες περιπτώσεις μη εφικτή την εξασφάλιση της μέγιστης αποδοτικότητας του παραλληλισμού.


\section{Το προγραμματιστικό μοντέλο του OpenMP}
Το OpenMP βασίζεται στη χρήση πολλαπλών νημάτων όπως άλλωστε συνηθίζεται στον προγραμματισμό συτημάτων κοινόχρηστης μνήμης καθώς και στο προγραμματιστικό μοντέλο fork-join που συναντάται και στις διεργασίες.

Στο μοντέλο fork-join (Σχήμα Χ), η εκτέλεση ξεκινάει σειριακά από ένα νήμα (γνωστό ως κύριο - master) και σε προκαθορισμένα σημεία όπου απαιτείται παράλληλη εκτέλεση, δημιουργούνται επιπλέον νήματα τα οποία μαζί με το κύριο νήμα συμμετέχουν στον παράλληλο υπολογισμό. Τα σημεία στα οποία πραγματοποιείται παράλληλη εκτέλεση είναι γνωστά ως παράλληλες περιοχές (parallel sections/regions). Μόλις ο παράλληλος υπολογισμός τελειώσει τα νήματα που δημιουργήθηκαν τερματίζουν και η εκτέλεση συνεχίζεται από το κύριο νήμα.

Ενδιαφέρον είναι το γεγονός ότι υποστηρίζονται αυθαίρετα πολλές εμφωλευμένες παράλληλες περιοχές, δηλαδή ένα οποιοδήποτε νήμα το οποίο συμμετέχει στην εκτέλεση μιας παράλληλης περιοχής μπορεί να αποτελέσει με τη σειρά του κύριο νήμα και να δημιουργήσει μία νέα παράλληλη περιοχή. Οι εμφωλευμένες παράλληλες περιοχές μπορούν να χρησιμοποιηθούν για την ανάθεση εργαστιών (tasks) επιθυμητού μεγέθους κόκκου παραλληλίας (granularity) σε κάθε νήμα.

\section{Εισαγωγή στη διεπαφή προγραμματισμού OpenMP}
\label{sec:Introduction to OpenMP API}
Η διπλωματική εργασία περιέχει $\nu$ κεφάλαια.

\section{Μεταφραστές OpenMP}
\label{sec:OpenMP Compilers}
Η διπλωματική εργασία περιέχει $\nu$ κεφάλαια.

