\chapter{Η διεπαφή προγραμματισμού OpenMP}
\label{ch:OpenMP API}

\section{Εισαγωγή στο OpenMP}
\label{sec:Introduction to OpenMP}
Όπως αναφέρθηκε ήδη στην υποενότητα Χ, η διεπαφή προγραμματισμού εφαρμογών OpenMP (Open Multi-Processing) αναπτύχθηκε για τη διευκόλυνση της ανάπτυξης πολυνηματικών εφαρμογών για συστήματα κοινόχρηστης μνήμης. Οι γλώσσες οι οποίες υποστηρίζονται είναι οι C, C++ και Fortran.
To OpenMP αποτελείται από:
\begin{itemize}
	\item Oδηγίες (directives): Συνιστούν οδηγίες προς τον μεταφραστή για το πώς και τι να εκτελέσει πολυνηματικά. Στις γλώσσες C/C++ χρησιμοποιείται ο μηχανισμός που είναι γνωστός ως pragmas και απευθύνεται στον προεπεξεργαστή. Αυτές οι οδηγίες προστίθενται στο υπάρχον σειριακό πρόγραμμα και μπορούν να αγνοηθούν από έναν μεταφραστή που δεν τις υποστηρίζει. Αυτό είναι μεγάλο προσόν καθώς το ίδιο πρόγραμμα μπορεί να εκτελεστεί σειριακά ή παράλληλα.
	\item Ρουτίνες βιβλιοθήκης: σύνολο συναρτήσεων οι οποίες βοηθούν στη διαχείριση των χαρακτηριστικών των νημάτων και του περιβάλλοντος εκτέλεσης. Για παράδειγμα, η συνάρτηση \texttt{omp\_set\_num\_threads(int)} καθορίζει το πλήθος των νημάτων που θα συμμετάσχουν σε επερχόμενη πολυνηματική εκτέλεση (παράλληλη περιοχή).
	\item Μεταβλητές περιβάλλοντος: χρησιμοποιούνται για τον καθορισμό διάφορων χαρακτηριστικών των νημάτων και του περιβάλλοντος εκτέλεσης. Οι τιμές των μεταβλητών περιβάλλοντος οριστικοποιούνται στην αρχή της εκτέλεσης και χρησιμοποιούνται ως προκαθορισμένες τιμές. Κάποιες από αυτές τις προκαθορισμένες αυτές τιμές μπορούν να τροποποιηθούν σε χρόνο εκτέλεσης με χρήση των διαθέσιμων ρουτινών βιβλιοθήκης.
\end{itemize}

Από τη στιγμή που η παραλληλοποίηση ενός σειριακού προγράμματος μπορεί να γίνει με την απλή προσθήκη οδηγιών στο υπάρχοντα κώδικα, η διαδικασία της παραλληλοποίησης απλοποιείται σε μεγάλο βαθμό και μπορεί να γίνει σταδιακά (π.χ. παραλληλοποίηση ενός βρόγχου for τη φορά) και χωρίς τη χρήση διαφορετικής λογικής όπως για παράδειγμα θα γινόταν με χρήση των POSIX Threads. 

Ένα από τα σχετικά καινούρια χαρακτηριστικά του OpenMP είναι η δυνατότητα αποστολής κώδικα για εκτέλεση σε συσκευές όπως κάρτες γραφικών γενικού σκοπού (GPGPUs), συνεπεξεργαστές (coprocessors) ή διάφορων άλλων επιταχυντών. Το πλεονέκτημα είναι ότι ο προγραμματιστής δεν χρειάζεται να μάθει να προγραμματίζει σε γλώσσες προγραμματισμού χαμηλού επιπέδου όπως OpenCL, CUDA κλπ για να αξιοποιήσει την επεξεργαστική ισχύ των διαθέσιμων συσκευών. Αυτό το χαρακτηριστικό είναι ιδιαίτερα βοηθητικό σε συστήματα υπολογισμών υψηλών επιδόσεων (High Performance Computing - HPC) όπου υπάρχει η τάση να εξοπλίζεται ένα σύνολο των υπολογιστικών κόμβων με ισχυρούς επιταχυντές για την επίτευξη μεγαλύτερων επιδόσεων. Για παράδειγμα, ο υπερυπολογιστής Aris του Εθνικού Δικτύου Υποδομών και Έρευνας (ΕΔΥΤΕ - GRNET) διαθέτει πλην άλλων:
\begin{itemize}
	\item 18 κόμβους με 2 επεξεργαστές Intel Xeon E5-2660v3 και 2 συνεπεξεργαστές Intel Xeon Phi 7120P.
	\item 44 κόμβους με 2 επεξεργαστές Intel Xeon E5-2660v3 και 2 κάρτες γραφικών NVIDIA K40.
	\item 1 κόμβο με 2 επεξεργαστές Intel E5-2698v4 και 8 κάρτες γραφικών NVIDIA V100 για εκτέλεση προγραμμάτων μηχανικής μάθησης.
\end{itemize}

Λαμβάνοντας υπόψην ότι η διαχείριση των νημάτων μετατίθεται από τον προγραμματιστή στο μεταφραστή, ότι απλοποιούνται ουσιώδη ζητήματα ενός παράλληλου προγράμματος όπως η επίτευξη συγχρονισμού και αμοιβαίου αποκλεισμού, καθώς επίσης ότι μπορούν να αξιοποιηθούν επιταχυντές χωρίς γνώση του πως προγραμματίζονται, γίνεται εύκολα αντιληπτό ότι το OpenMP είναι ένα προσιτό εργαλείο ακόμα και για άτομα χωρίς μεγάλη εμπειρία στον παράλληλο προγραμματισμό. Αυτό το χαρακτηριστικό του OpenMP το κάνει ιδιαίτερα διαδεδομένο σε χρήστες που το υπόβαθρό τους διαφέρει από αυτό της επιστήμης της πληροφορικής, όπως για παράδειγμα φυσικοί, χημικοί, αστρονόμοι κόκ. Ταυτόχρονα όμως, η απόκρυψη των λεπτομερειών χαμηλού επιπέδου είναι πιθανό να καταστήσει σε ορισμένες περιπτώσεις μη εφικτή την εξασφάλιση της μέγιστης αποδοτικότητας του παραλληλισμού.

\section{Το προγραμματιστικό μοντέλο του OpenMP}
Το OpenMP βασίζεται στη χρήση πολλαπλών νημάτων όπως άλλωστε συνηθίζεται στον προγραμματισμό συτημάτων κοινόχρηστης μνήμης καθώς και στο προγραμματιστικό μοντέλο fork-join που συναντάται και στις διεργασίες.

Στο μοντέλο fork-join (Σχήμα Χ), η εκτέλεση ξεκινάει σειριακά από ένα νήμα (γνωστό ως αρχηγός - master) και σε προκαθορισμένα σημεία όπου απαιτείται παράλληλη εκτέλεση, δημιουργούνται επιπλέον νήματα τα οποία μαζί με το νήμα-αρχηγό συμμετέχουν στον παράλληλο υπολογισμό. Τα σημεία στα οποία πραγματοποιείται παράλληλη εκτέλεση είναι γνωστά ως παράλληλες περιοχές (parallel sections/regions). Μόλις ο παράλληλος υπολογισμός τελειώσει τα νήματα που δημιουργήθηκαν τερματίζουν και η εκτέλεση συνεχίζεται από το νήμα-αρχηγό.

Ενδιαφέρον είναι το γεγονός ότι υποστηρίζονται αυθαίρετα πολλές εμφωλευμένες παράλληλες περιοχές, δηλαδή ένα οποιοδήποτε νήμα το οποίο συμμετέχει στην εκτέλεση μιας παράλληλης περιοχής μπορεί να αποτελέσει με τη σειρά του νήμα-αρχηγός και να δημιουργήσει μία νέα παράλληλη περιοχή. Οι εμφωλευμένες παράλληλες περιοχές μπορούν να χρησιμοποιηθούν για την ανάθεση εργαστιών (tasks) επιθυμητού μεγέθους κόκκου παραλληλίας (granularity) σε κάθε νήμα.

Είναι χρήσιμο να αναφερθεί πως όταν ξεκινάει να εκτελείται μία διεργασία, χρησιμοποιείται ένα νήμα για την εκτέλεση των εντολών σειριακά και το οποίο νήμα στα πλαίσια του OpenMP ονομάζεται αρχικό νήμα (initial thread).

\section{Εισαγωγή στη διεπαφή προγραμματισμού OpenMP}

Το πλήθος των σελίδων των προδιαγραφών του OpenMP τείνει να αυξάνεται εκθετικά στις τελευταίες εκδόσεις με αποτέλεσμα να είναι αδύνατο να περιγραφούν όλες οι δυνατότητές του στα πλαίσια μίας διπλωματικής εργασίας. Για το λόγο αυτό, θα γίνει περιγραφή ενός υποσυνόλου των διαθέσιμων λειτουργιών, πολλές από τις οποίες αποτελούν τις πιο συνηθισμένες και καλύπτουν τις ανάγκες της πλειοψηφίας των διαθέσιμων εφαρμογών OpenMP. Αυτές οι πιο διαδεδομένες λειτουργίες είναι εικοσιμία (21) στο πλήθος και αποτελούν το λεγόμενο OpenMP Common Core @Ref.

\subsection{Σύνταξη οδηγιών (directives)}
Η γενική μορφή μίας οδηγίας στο OpenMP είναι της μορφής:

\begin{quote}
	\texttt{\textbf{\#pragma omp} \textit{directive-name [[,] clause [[,] clause] ... ] <new-line>}}
\end{quote}

\noindent Κάθε οδηγία ξεκινάει υποχρεωτικά με το \texttt{\#pragma omp} και ακολουθεί το όνομα της οδηγίας (directive) που καθορίζει ποιά λειτουργία θα εκτελεστεί στην περιοχή του κώδικα που ακολουθεί. Υπάρχουν διάφορες διαθέσιμες οδηγίες οι οποίες μπορούν να χρησιμοποιηθούν πλην άλλων για τη δημιουργία παράλληλης ομάδας, το συγχρονισμό των νημάτων και τον ορισμό της πολιτικής με την οποία τα νήματα θα ανατεθούν στους διαθέσιμους επεξεργαστές.

Στη συνέχεια τοποθετούνται προαιρετικά φράσεις (clauses) οι οποίες παραμετροποιούν τις συνθήκες υπό τις οποίες θα εκτελεστεί η λειτουργία που ορίζει η οδηγία. Για παράδειγμα, η φράση \texttt{num\_threads} μπορεί να χρησιμοποιηθεί σε συνδιασμό με την οδηγία δημιουργίας παράλληλης ομάδας για να καθορίσει το επιθυμητό πλήθος των νημάτων που θα συμμετάσχουν στην παράλληλη εκτέλεση. Η σειρά με την οποία αναγράφονται οι φράσεις δεν έχει σημασία.

Το τέλος της οδηγίας σηματοδοτείται από αλλαγή γραμμής (newline).

\subsection{Η οδηγία parallel}
Η οδηγία \texttt{parallel} συντάσσεται ως εξής:

\begin{quote}
	\texttt{\textbf{\#pragma omp parallel} \textit{[clause [[,] clause] ... ] <new-line>}} \\
		\texttt{\textit{structured-block}}
\end{quote}

Όταν ένα οποιοδήποτε νήμα συναντήσει μία οδηγία \texttt{parallel}, δημιουργείται μία ομάδα νημάτων η οποία εκτελεί την παράλληλη περιοχή. Μια παράλληλη περιοχή υποδηλώνει ένα τμήμα κώδικα το οποίο προορίζεται για πολυνηματική εκτέλεση. Στη συγκεκριμένη περίπτωση, ο κώδικας που περιέχεται στο δομημένο τμήμα κώδικα (structured block) που ακολουθεί την οδηγία \texttt{parallel} είναι αυτός που εν τέλει θα εκτελεστεί παράλληλα. Ώς δομημένο τμήμα κώδικα ορίζεται μία εντολή ή μία ακολουθία εντολών που περικλείονται από άγκιστρα.

Το πλήθος των νημάτων ($N$) που συμμετέχουν σε μια παράλληλη ομάδα είναι σταθερό καθόλη τη διάρκεια της. Το νήμα το οποίο συνάντησε την οδηγία \texttt{parallel} λαμβάνει το ρόλο του αρχηγού (master\footnote{Στο πρότυπο του OpenMP 5.1 (Νοέμβριος 2020) καθορίζεται αλλαγή της ονομασίας master σε primary. Στο παρόν κείμενο θα ακολουθηθεί η ορολογία master (αρχηγός) για λόγους συμβατότητας με τους υπάρχοντες πόρους της Ομάδας Παράλληλης Επεξεργασίας του Πανεπιστημίου Ιωαννίνων.}) της ομάδας, ενώ συμμετέχει και αυτό στον παράλληλο υπολογισμό μαζί με τα υπόλοιπα $N-1$ νήματα που δημιουργήθηκαν.

Μέσα σε μία παράλληλη περιοχή μπορούν να χρησιμοποιηθούν τα αναγνωριστικά των νημάτων για την αναγνώριση του κάθε νήματος. Τα αναγνωριστικά είναι ακέραιοι αριθμοί και συγκεκριμένα για μία ομάδα $N$ νημάτων, η τιμή τους κυμαίνεται από μηδέν (για τον αρχηγό της ομάδας) έως και ένα λιγότερο από το μέγεθος της ομάδας, δηλαδή $N-1$.

Καθώς ο χρόνος που χρειάζεται ένα νήμα για να εκτελέσει την παράλληλη περιοχή εξαρτάται από πολλούς παράγοντες, όπως για παράδειγμα το πόσο δίκαιη είναι η κατανομή του φόρτου μεταξύ των νημάτων της ομάδας, στο τέλος της παράλληλης περιοχής υπονοείται μία κλήση φραγής (barrier). Η κλήση αυτή εξασφαλίζει ότι τα νήματα θα περιμένουν στην κλήση φραγής μέχρις ότου όλα τα νήματα να φτάσουν σε αυτό το σημείο πριν τους επιτραπεί να τερματίσουν και το νήμα-αρχηγός συνεχίσει την εκτέλεση του υπόλοιπου προγράμματος που ακολουθεί της παράλληλης περιοχής.

Την οδηγία \texttt{parallel} μπορεί προεραιτικά να ακολουθούν φράσεις διαμοιρασμού δεδομένων που θα δούμε στην υποενότητα Χ @Ref καθώς και οι φράσεις \texttt{num\_threads}, \texttt{redution} και \texttt{proc\_bind} που περιγράφονται αμέσως μετά.

\subsubsection{Η φράση num\_threads}
Η φράση \texttt{num\_threads} δέχεται ως παράμετρο έναν ακέραιο αριθμό και καθορίζει το επιθυμητό πλήθος των νημάτων που θα εκτελέσουν την παράλληλη περιοχή.

\subsubsection{Η φράση reduction}
Η φράση \texttt{reduction} χρησιμοποιείται για τη διεκπεραίωση μιας αριθμητικής πράξης από τα νήματα της ομάδας πάνω σε μια κοινόχρηστη μεταβλητή, εξασφαλίζοντας την αποφυγή συνθηκών ανταγωνισμού. Η απλουστευμένη σύνταξη της φράσης είναι η εξής:
\begin{quote}
	\texttt{\textbf{reduction(}\textit{reduction-identifier : list}\textbf{)}}
\end{quote}

Ο \texttt{reduction-identifier} είναι η πράξη η οποία θέλουμε να εφαρμοστεί πάνω στις μία ή περισσότερες μεταβλητές που αναγράφονται στη λίστα \texttt{list} και είναι διαχωρισμένες μεταξύ τους με κόμμα. Οι διαθέσιμες πράξεις είναι οι \texttt{+}, \texttt{-}, \texttt{*}, \texttt{\&}, \texttt{\textbar}, \texttt{\string^}, \texttt{\&\&} και \texttt{\textbar\textbar}.

\subsubsection{Η φράση proc\_bind}
Η φράση \texttt{proc\_bind} χρησιμοποιείται για τον ορισμό της πολιτικής με την οποία τα νήματα θα ανατεθούν στους διαθέσιμους επεξεργαστές. Οι διαθέσιμες επιλογές είναι οι \texttt{master/primary}, \texttt{close}, \texttt{spread}. Περισσότερες λεπτομέρειες θα δούμε στην Υποενότητα Χ όπου θα ασχοληθούμε λεπτομερώς με την τοπολογία του υποκείμενου συστήματος και τον τρόπο ανάθεσης των νημάτων OpenMP στους επεξεργαστές.


\subsection{Φράσεις διαμοιρασμού δεδομένων}
Οι φράσεις διαμοιρασμού δεδομένων χρησιμοποιούνται για να δηλώσουν τον τρόπο διαμοιρασμού μίας ή περισσότερων μεταβλητών μεταξύ των νημάτων ως εξής:
\begin{itemize}
	\item \texttt{shared}: Οι μεταβλητές είναι κοινόχρηστες.
	\item \texttt{private}: Οι μεταβλητές είναι ιδιωτικές καθώς δημιουργείται από ένα αντίγραφο για κάθε νήμα.
	\item \texttt{firstprivate}: Οι μεταβλητές είναι ιδιωτικές και καθεμιά αρχικοποιείται στην τιμή της αντίστοιχης αρχικής μεταβλητής.
	\item \texttt{lastprivate}: Οι μεταβλητές είναι ιδιωτικές και μετά το πέρας της εκτέλεσης η τιμή της καθεμιάς τους θα χρησιμοποιηθεί για την ενημέρωση της τιμής της αντίστοιχης αρχικής μεταβλητής.
	\item \texttt{default}: Καθορίζει την προκαθορισμένη πολιτική διαμοιρασμού με διαθέσιμες επιλογές να είναι οι \texttt{shared}, \texttt{firstprivate}, \texttt{private}, \texttt{none}.
\end{itemize}


\subsection{Οδηγίες Διαμοιρασμού Εργασίας}
Το OpenMP παρέχει τη δυνατότητα κατανομής του φόρτου εργασίας ανάμεσα στα νήματα της ομάδας μέσω των οδηγιών που καθορίζουν περιοχές διαμοιρασμού εργασίας (worksharing regions).

Η βασική διαφορά μίας περιοχής διαμοιρασμού εργασίας με μία παράλληλη περιοχή είναι ότι στην πρώτη σε αντίθεση με τη δεύτερη, δεν δημιουργούνται νέα νήματα αλλά χρησιμοποιούνται τα υπάρχοντα. Βάσει αυτής της παρατήρησης συμπεραίνουμε ότι οι περιοχές διαμοιρασμού έχουν νόημα όταν εντοπίζονται εντός παράλληλων περιοχών. Είναι πιθανό μία παράλληλη ομάδα που αποτελείται από ένα μόνο νήμα να συναντήσει μία οδηγία περιοχής διαμοιρασμού εργασίας. Όπως είναι προφανές, σε αυτή την περίπτωση, η εκτέλεση είναι σειριακή και όχι παράλληλη.

Στο τέλος των περιοχών διαμοιρασμού εργασίας, όπως και στο τέλος των παράλληλων περιοχών, υπονοείται μία κλήση φραγής για την επίτευξη συγχρονισμού μεταξύ των νημάτων, με τη διαφορά ότι στις πρώτες η κλήση φραγής μπορεί να παραληφθεί με τη χρήση της φράσης \texttt{nowait}.

\subsubsection{Οδηγία sections}
Χρησιμοποιείται για την κατανομή μη επαναληπτικών (non-iterative) εργασιών και συντάσσεται ως εξής:
\begin{quote}
	\texttt{\textbf{\#pragma omp sections} \textit{[clause [[,] clause] ... ] <new-line>}} \\
	\texttt{\{} \\
		\texttt{\textit{[} \textbf{\#pragma omp section} \textit{<new-line>}} \\
		\texttt{\textit{<structured-block> ]}} \\
		\texttt{\textit{[} \textbf{\#pragma omp section} \textit{<new-line>}} \\
		\texttt{\textit{<structured-block> ]}} \\
		... \\
	\texttt{\}}
\end{quote}

Κάθε δομημένο τμήμα κώδικα που ακολουθεί μία οδηγία \texttt{\#pragma omp section} που περιέχεται μέσα στην οδηγία \texttt{sections} θα ανατεθεί σε ένα νήμα και θα εκτελεστεί ακριβώς μία φορά. Συνήθεις φράσεις αποτελούν οι \texttt{private}, \texttt{firstprivate}, \texttt{lastprivate} και \texttt{reduction}.

\subsubsection{Οδηγία single}
Η οδηγία αυτή καθορίζει ότι το δομημένο τμήμα κώδικα που την ακολουθεί θα εκτελεστεί μόνο από ένα νήμα (όχι απαραίτητα το νήμα-αρχηγό) και η σύνταξή της είναι η εξής:
\begin{quote}
	\texttt{\textbf{\#pragma omp single} \textit{[clause [[,] clause] ... ] <new-line>}} \\
		\texttt{\textit{<structured-block>}}
\end{quote}

Συνήθεις φράσεις που ακολουθούν είναι οι \texttt{private} και \texttt{firstprivate}.


\subsection{Οδηγίες Διαμοιρασμού Εργασίας Βρόγχου}
Οι οδηγίες διαμοιαρμού εργασίας βρόγχου είναι αντίστοιχες με τις οδηγίες διαμοιρασμού εργασίας που είδαμε στην υποενότητα Χ @Ref, με τη διαφορά ότι αφορούν την κατανομή των επαναλήψεων ενός βρόγχου στα νήματα.

\subsubsection{Οδηγία for}
Χρησιμοποιείται για την κατανομή των επαναλήψεων ενός βρόγχου\footnote{Η σύνταξη του βρόγχου δεν μπορεί να είναι τόσο αυθαίρετη όσο επιτρέπει το συντακτικό των γλωσσών C/C++, αλλά στα πλαίσια αυτής της εργασίας δεν θα μας απασχολήσει αυτό το ζήτημα.} \texttt{for} και συντάσσεται ως εξής:
\begin{quote}
	\texttt{\textbf{\#pragma omp for} \textit{[clause [[,] clause] ... ] <new-line>}} \\
		\texttt{\textit{<loop-nest>}}
\end{quote}

Την οδηγία αυτή ακολουθεί υποχρεωτικά βρόγχος \texttt{for} ενώ συνήθεις φράσεις αποτελούν οι \texttt{private}, \texttt{firstprivate}, \texttt{lastprivate}, \texttt{reduction} και \texttt{schedule}.

Η φράση \texttt{schedule} χρησιμοποιείται για τον καθορισμό της πολιτικής με την οποία θα διαμοιραστούν οι επαναλήψεις ενός βρόγχου \texttt{for} στα νήματα και η απλουστευμένη σύνταξη της είναι η ακόλουθη:
\begin{quote}
	\texttt{\textbf{schedule(}\textit{kind[, chunk\_size]}\textbf{)}}
\end{quote}

Η τιμή \texttt{kind} καθορίζει τον τρόπο διαμοιρασμού των επαναλήψεων ενώ η τιμή \texttt{chunk\_size} καθορίζει το μέγεθος του κόκκου παραλληλίας (granularity) που αναλαμβάνει το κάθε νήμα. Η χρονοδρομολόγηση των επαναλήψεων γίνεται όπως περιγράφεται ακολούθως:
\begin{itemize}
	\item \texttt{static}: Συνεχόμενες επαναλήψεις διασπώνται σε τμήματα μεγέθους όσο η τιμή \texttt{chunk\_size} και ανατίθενται κυκλικά (round-robin) στα νήματα βάσει του αναγνωριστικού τους. Σε περίπτωση που δεν έχει καθοριστεί συγκεκριμένη τιμή \texttt{chunk\_size}, το σύνολο των επαναλήψεων χωρίζεται σε τόσα ισομεγέθη τμήματα όσα και το πλήθος των νημάτων.
	\item \texttt{dynamic}: Συνεχόμενες επαναλήψεις διασπώνται σε τμήματα μεγέθους όσο η τιμή \texttt{chunk\_size} και ανατίθενται σε νήματα όταν αυτά ζητήσουν το επόμενο τμήμα προς εκτέλεση. Η δυναμική ανάθεση συνεχίζεται μέχρι να ανατεθούν όλα τα τμήματα. Σε περίπτωση που δεν έχει καθοριστεί συγκεκριμένη τιμή \texttt{chunk\_size}, τότε τα τμήματα έχουν μέγεθος ίσο με 1.
	\item \texttt{guided}: Η πολιτική αυτή μοιάζει στην πολιτική \texttt{dynamic} με τη διαφορά ότι το μέγεθος των τμημάτων δεν είναι σταθερό. Συγκεκριμένα, για \texttt{chunk\_size} ίσο με 1 ($k$), το μέγεθος του πρώτου τμήματος ισούται με το πλήθος όλων των επαναλήψεων διαιρεμένο με το πλήθος των νημάτων, με το μέγεθος των επόμενων τμημάτων να μειώνεται εκθετικά μέχρι να ισούται με 1 ($k$).
	\item \texttt{auto}: Η επιλογή πολιτικής και μεγέθους κόκκου παραλληλίας μεταφέρεται από τον προγραμματιστή στο μεταφραστή ή στο σύστημα χρόνου εκτέλεσης (runtime).
	\item \texttt{runtime}: Η πολιτική και το μέγεθος κόκκου παραλληλίας καθορίζονται σε χρόνο εκτέλεσης βάσει της τιμής της μεταβλητής περιβάλλοντος \texttt{OMP\_SCHEDULE}.
\end{itemize}

Στην περίπτωση των πολιτικών \texttt{auto} και \texttt{runtime} απαγορεύεται ο προσδιορισμός τιμής \texttt{chunk\_size}.

\subsection{Η οδηγία task}
Η οδηγία αυτή ορίζει μία εργασία προς εκτέλεση και συντάσσεται ως εξής:
\begin{quote}
	\texttt{\textbf{\#pragma omp task} \textit{[clause [[,] clause] ... ] <new-line>}} \\
		\texttt{\textit{<structured-block>}}
\end{quote}

Όταν ένα νήμα συναντήσει την οδηγία \texttt{task}, δημιουργεί μία νέα εργασία που αποτελείται από τον κώδικα του δομημένου τμήματος κώδικας και το περιβάλλον δεδομένων (data environment) που χρειάζεται η εργασία για να διεκπεραιωθεί. Η εκτέλεση της εργασίας μπορεί να πραγματοποιηθεί από οποιοδήποτε νήμα και δεν είναι γνωστό πότε θα ξεκινήσει. Να σημειωθεί ότι υποστηρίζονται εμφωλευμένες οδηγίες \texttt{task}.

Συνήθεις φράσεις αποτελούν οι \texttt{default}, \texttt{private}, \texttt{firstprivate} και \texttt{shared}.


\subsection{Οδηγίες συγχρονισμού}
Στα προγράμματα OpenMP ως προγράμματα για συστήματα κοινόχρηστης μνήμης, σημαντικός είναι ο ρόλος του συγχρονισμού για την εξασφάλιση της συνέπειας των δεδομένων και της ορθότητας του προγράμματος.

\subsubsection{Οδηγία atomic}
Απλουστευμένη σύνταξη:
\begin{quote}
	\texttt{\textbf{\#pragma omp atomic} \textit{<new-line>}} \\
		\texttt{\textit{<statement>}}
\end{quote}

Εξασφαλίζει ότι μια θέση μνήμης προσβαίνεται ατομικά εξαλείφοντας την πιθανότητα πολλαπλών ταυτόχρονων προσβάσεων από διαφορετικά νήματα.

\subsubsection{Οδηγία barrier}
Σύνταξη:
\begin{quote}
	\texttt{\textbf{\#pragma omp barrier} \textit{<new-line>}}
\end{quote}

Η γνωστή κλήση φραγής η οποία εξασφαλίζει ότι όλα τα νήματα της ομάδας θα περιμένουν σε αυτό το σημείο πριν μπορέσουν να συνεχίσουν την εκτέλεση με τις εντολές που ακολουθούν.

\subsubsection{Οδηγία critical}
Απλουστευμένη σύνταξη:
\begin{quote}
	\texttt{\textbf{\#pragma omp critical} \textit{<new-line>}} \\
		\texttt{\textit{<structured-block>}}
\end{quote}

Εξασφαλίζει ότι το δομημένο τμήμα κώδικα που ακολουθεί θα εκτελείται από ένα νήμα μόνο τη φορά.

\subsubsection{Οδηγία taskwait}
Απλουστευμένη σύνταξη:
\begin{quote}
	\texttt{\textbf{\#pragma omp taskwait} \textit{new-line>}}
\end{quote}

Εξασφαλίζει ότι οι εργασίες-παιδιά της τρέχουσας εργασίας (task) θα έχουν ολοκληρωθεί πριν συνεχιστεί η εκτέλεση μετά από αυτό το σημείο.

\subsection{Ρουτίνες βιβλιοθήκης χρόνου εκτέλεσης}
\begin{itemize}
	\item \texttt{void omp\_set\_num\_threads(int num\_threads)}: Θέτει την τιμή της παραμέτρου \texttt{num\_threads} ως το πλήθος των νημάτων που θα χρησιμοποιηθούν σε επερχόμενες παράλληλες περιοχές. Εξαίρεση αποτελούν οι παράλληλες περιοχές που χρησιμοποιούν τη φράση \texttt{num\_threads} η οποία υπερισχύει.
	\item \texttt{int omp\_get\_num\_threads(void)}: Επιστρέφει το πλήθος των νημάτων που συνιστούν την τρέχουσα ομάδα.
	\item \texttt{int omp\_get\_thread\_num(void)}: Επιστρέφει το αριθμητικό αναγνωριστικό του καλούντος νήματος μέσα στην τρέχουσα ομάδα.
	\item \texttt{double omp\_get\_wtime(void)}: Επιστρέφει το χρόνο (wall clock) σε δευτερόλεπτα που παρήλθε μετά από μία δεδομένη στιγμή στο παρελθόν.
\end{itemize}

\subsection{Μεταβλητές περιβάλλοντος}
Η μεταβλητή περιβάλλοντος \texttt{OMP\_NUM\_THREADS} μπορεί να χρησιμοποιηθεί για τον ορισμό ενός προκαθορισμένου πλήθους νημάτων που θα συμμετέχουν στις παράλληλες περιοχές του προγράμματος. Η τελική απόφαση για το πλήθος των νημάτων που θα χρησιμοποιηθούν σε μία παράλληλη περιοχή καθορίζεται σε φθίνουσα προτεραιότητα από:
\begin{enumerate}
	\item Τη μεταβλητή περιβάλλοντος \texttt{OMP\_NUM\_THREADS}
	\item Τη ρουτίνα βιβλιοθήκης χρόνου εκτέλεσης \texttt{omp\_set\_num\_threads}
	\item Τη φράση \texttt{num\_threads} της οδηγίας \texttt{\#pragma omp parallel}
\end{enumerate}


\section{Μεταφραστές OpenMP}
Η διεπαφή που ορίζεται από το πρότυπο του OpenMP υλοποιείται από διάφορους εμπορικούς και ερευνητικούς μεταφραστές. Προμηθευτές μεταφραστών για C/C++ που υποστηρίζουν το OpenMP είναι οι εξής:

\begin{itemize}
	\item AMD: 
	\begin{itemize}
		\item Ο AOMP είναι βασισμένος στον LLVM/Clang και υποστηρίζει την εκτέλεση κώδικα σε πολλαπλές κάρτες γραφικών/επιταχυντές.
		\item Ο AOCC είναι επίσης βασισμένος στον clang/LLVM και υποστηρίζει πλήρως το OpenMP 4.5 και μερικώς το OpenMP 5.0.
	\end{itemize}
	\item ARM: Ο μεταφραστής της ARM παρέχει πλήρη υποστήριξη για το OpenMP 3.1 και υποστηρίζει το OpenMP 4.0/4.5 χωρίς τη δυνατότητα εκτέλεσης κώδικα σε συσκευές η οποία βρίσκεται υπό ανάπτυξη.
	\item Barcelona Supercomputing Center: Ο Mercurium είναι ένας ερευνητικός μεταφραστής πηγαίου σε πηγαίο κώδικα (source-to-source) ο οποίος υποστηρίζει σχεδόν πλήρως το OpenMP 3.1 καθώς και χαρακτηριστικά νεότερων εκδόσεων που σχετίζονται με τον μηχανισμό tasking του OpenMP.
	\item Fujitsu: Οι μεταφραστές για τον υπερυπολογιστή PRIMEHPC FX100 της Fujitsu υποστηρίζουν το OpenMP 3.1.
	\item GNU: Ο GCC υποστηρίζει πλήρως το OpenMP 4.5 (έκδοση 6) και μερικώς το OpenMP 5.0. Επίσης, σε συστήματα Linux υποστηρίζει την εκτέλεση κώδικα σε κάρτες γραφικών NVIDIA (nvptx) και τις κάρτες Fiji και Vega της AMD Radeon (GCN).
	\item HPE: Το Cray Compiling Environment (CCE) παρέχει πλήρη υποστήριξη για το OpenMP 4.5 και μερική υποστήριξη για το OpenMP 5.0.
	\item IBM: Ο μεταφραστής XL C/C++ για Linux υποστηρίζει πλήρως το OpenMP 4.5.
	\item Intel: Οι μεταφραστές της Intel υποστηρίζουν πλήρως το OpenMP 4.5 και μερικώς το OpenMP 5.0.
	\item LLNL Rose Research Compiler: Ο ROSE είναι ένας ερευνητικός μετφραστής πηγαίου σε πηγαίο κώδικα που υποστηρίζει το OpenMP 3.0 και κάποια χαρακτηριστικά του OpenMP 4.0 που σχετίζονται με την εκτέλεση κώδικα σε κάρτες γραφικών/επιταχυντές της NVIDIA.
	\item LLVM: Ο Clang παρέχει υποστήριξη για το OpenMP 4.5 με περιορισμένη υποστήριξη για την εκτέλεση κώδικα σε συσκευές. Επίσης, υποστηρίζεται μεγάλο μέρος του OpenMP 5.0 και μικρό μέρος του OpenMP 5.1.
	\item Siemens: Ο Sourcery CodeBench (AMD GCN) Lite για συστήματα x86\_64 GNU/Linux είναι βασισμένος στον GCC, παρέχει πλήρη υποστήριξη για το OpenMP 4.5, μερική υποστήριξη για το OpenMP 5.0. και επιτρέπει την εκτέλεση κώδικα σε κάρτες γραφικών AMD Radeon (GCN) όπως οι Fiji, gfx900 Vega 10 και gfx906 Vega 20.	
	\item NVIDIA HPC Compiler: Οι μεταφραστές NVIDIA HPC παρέχουν πλήρη υποστήριξη του OpenMP 3.1 και μερική υποστήριξη του OpenMP 5.0 για συστήματα Linux/x86-64, Linux/OpenPOWER, Linux/Arm. Επίσης, σε κάρτες γραφικών της NVIDIA υποστηρίζεται μερικώς το OpenMP 5.0.
	\item OpenUH Research Compiler: Ο OpenUH είναι ερευνητικός μεταφραστής και υποστηρίζει πλήρως το OpenMP 2.5 και σχεδόν πλήρως το OpenMP 3.0 σε συστήματα Linux.
	\item Oracle: Οι μεταφραστές του Oracle Developer Studio υποστηρίζουν το OpenMP 4.0.
	\item PGI: Οι μεταφραστές NVidia HPC υποστηρίζουν μερικώς το OpenMP 5.0.
	\item Texas Instruments:
		\begin{itemize}
			\item Ο μεταφραστής TI cl6x υποστηρίζει το OpenMP 3.0. (C66x).
			\item Το βασισμένο στον GCC Linaro toolchain υποστηρίζει το OpenMP 4.5 (Cortex-A15).
			\item Ο μεταφραστής TI clacc υποστηρίζει το OpenMP 3.0 και εκτέλεση κώδικα σε συσκευές βάσει του OpenMP 4.0 (Cortex-A15+C66x-DSP).
		\end{itemize}
		Να σημειωθεί ότι η παρεχόμενη υποτήριξη αφορά προϊόντα system on a chip (SoC) της Texas Instruments.
\end{itemize}

\subsection{Ο μεταφραστής OMPi}

