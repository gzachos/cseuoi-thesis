\chapter{Συγχρονισμός με barriers}
\label{ch:Synchronization with Barriers}
Όπως έχουμε ήδη αναφέρει σε προηγούμενα κεφάλαια, ο προγραμματισμός συστημάτων κοινόχρηστου χώρου διευθύνσεων βασίζεται συνήθως στη χρήση νημάτων, ο συγχρονισμός μεταξύ των οποίων είναι ιδιαίτερα σημαντικός καθώς εξασφαλίζει τη συνέπεια των δεδομένων και την ορθότητα του προγράμματος.

Μία από τις πιο διαδεδομένες μεθόδους συγχρονισμού είναι η \textit{κλήση φραγής}, γνωστή και ως barrier. Όταν στον πηγαίο κώδικα υπάρχει μία κλήση φραγής και συναντηθεί από ένα νήμα, τότε το νήμα περιμένει σε αυτό το σημείο την άφιξη των υπόλοιπων νημάτων, πριν μπορέσουν όλα μαζί να συνεχίσουν την εκτέλεση του κώδικα που ακολουθεί την κλήση φραγής.

%Οι κλήσεις φραγής είναι ιδιαίτερα χρήσιμες για τον διαχωρισμό των φάσεων ενός προγράμματος. Για παράδειγμα, αν θέλουμε να παραλληλοποιήσουμε ένα πρόγραμμα το οποίο υπολογίζει με κάποιο τρόπο τα στοιχεία ενός διδιάστατου πίνακα $A$ (1\textsuperscript{η} φάση) και στη συνέχεια αυτός ο πίνακας χρησιμοποιείται για τον υπολογισμό ενός άλλου πίνακα $B$ (2\textsuperscript{η} φάση), μία κλήση φραγής ανάμεσα στις δύο φάσεις θα εξασφαλίσει ότι πρώτα θα ολοκληρωθεί ο υπολογισμός ολόκληρου του πίνακα $Α$ και μετά θα συνεχίσουν τα νήματα στη δεύτερη φάση. 

Οι κλήσεις φραγής είναι ιδιαίτερα χρήσιμες για τον διαχωρισμό των φάσεων ενός προγράμματος. Για παράδειγμα, αν θέλουμε να παραλληλοποιήσουμε ένα πρόγραμμα το οποίο υλοποιεί μία επαναληπτική μέθοδο (π.χ. Conjugate gradient method), θα πρέπει να χρησιμοποιήσουμε κλήση φραγής ανάμεσα στις επαναλήψεις $n$ και $n+1$ ώστε να μην προχωρήσει κάποιο νήμα στην επανάληψη $n+1$ ενώ δεν έχει ολοκληρωθεί ο υπολογισμός της επανάληψης $n$.

Οι κλήσεις φραγής εκτός από το συγχρονισμόυ νημάτων μπορούν να χρησιμοποιηθούν και για το συγχρονισμό διεργασιών, κάτι που όμως δεν θα μας απασχολήσει στο πλαίσιο της παρούσας διπλωματικής εργασίας.

\section{Υλοποιήσεις barrier}
\label{sec:Barrier Implementations}
Παρόλο που η λογική των κλήσεων φραγής είναι αρκετά απλή, υπάρχει πληθώρα αλγορίθμων για την υλοποίησή τους. 

Μία πρώτη κατηγορία κλήσεων φραγής αποτελούν οι κεντρικοποιημένες κλήσεις φραγής (centralized barriers). Κάθε νήμα ενημερώνει μία κοινόχρηστη κατάσταση, ώστε να σημάνει την άφιξη του και έπειτα ελέγχει συνεχώς αυτή την κατάσταση μέχρι να αντιληφθεί ότι έφτασαν όλα τα νήματα. Μόλις φτάσουν όλα τα νήματα, το νήμα συνεχίζει την εκτέλεση του κώδικα που ακολουθεί μετά την κλήση φραγής. Στην πιο απλή της μορφή, η κοινόχρηστη κατάσταση μπορεί να είναι μερικές κοινόχρηστες μεταβλητές που χρησιμοποιούνται για παράδειγμα για τη μέτρηση του πλήθους των νημάτων που αφίχθησαν στον barrier.

Καθώς οι κλήσεις φραγής χρησιμοποιούνται επαναληπτικά, για παράδειγμα όταν υπάρχει ένας barrier στο τέλος ενός βρόγχου \texttt{for}, κατά τη διάρκεια μίας κλήσης τα νήματα θα πρέπει να πραγματοποιήσουν δύο ελέγχους. Έναν έλεγχο που θα εξασφαλίζει ότι όλα τα νήματα έφυγαν από την προηγούμενη κλήση φραγής και έναν έλεγχο που θα εξασφαλίζει ότι όλα τα νήματα έφτασαν στην τρέχουσα κλήση φραγής. Ο έλεγχος για το αν τα νήματα έφυγαν από την προηγούμενη κλήση φραγής είναι απαραίτητος καθώς σε αντίθετη περίπτωση μπορεί να δημιουργηθεί πρόβλημα στο διαχωρισμό δύο διαδοχικών κλήσεων φραγής από κάποιο γνήσιο υποσύνολο των νημάτων. Πιο συγκεκριμένα, υπάρχει η πιθανότητα ένα ή περισσότερα νήματα να εγκλωβιστούν στην προηγούμενη κλήση φραγής, την ώρα που κάποια άλλα έχουν μεταβεί στην επόμενη και ως συνέπεια το πρόγραμμα να οδηγηθεί σε αδιέξοδο (deadlock).

Η λύση για την αποφυγή του διπλού ελέγχου χωρίς όμως την δυνατότητα οδήγησης σε αδιέξοδο είναι η εισαγωγή της πληροφορίας \texttt{sense} η οποία παίρνει τιμές \texttt{true} (1) ή \texttt{false} (0). Η πληροφορία \texttt{sense} είναι κοινόχρηστη και η τιμή της αντιστρέφεται μεταξύ δύο διαδοχικών χρήσεων του barrier. Λόγω αυτής της αντιστροφής, η τιμή \texttt{sense} χρησιμοποιείται για το διαχωρισμό μεταξύ διαδοχικών χρήσεων ενός barrier. Οι κλήσεις φραγής που χρησιμοποιούν αυτή την τεχνική είναι γνωστές ως \texttt{sense reversing barriers}.

Το μειονέκτημα των κεντρικοποιημένων κλήσεων φραγής είναι οι συνεχόμενες προσβάσεις από όλα τα νήματα σε μία μόνο κοινόχρηστη τοποθεσία. Το πρόβλημα αυτό οδήγησε στην πρόταση υλοποιήσεων που χρησιμοποιούν μεθόδους οπισθοδρόμησης (backoff), όπως για παράδειγμα τη μέθοδο της εκθετικής οπισθοδρόμησης (exponential backoff).  Αν και με τη χρήση οπισθοδρόμησης παρατηρήθηκε ουσιαστική μείωση της κίνησης στο δίκτυο διασύνδεσης\footnote{Παρατηρήθηκε ταυτόχρονη αύξηση της καθυστέρησης του barrier, δηλαδή του χρόνου μεταξύ της τελευταίας άφιξης και της τελευταίας αναχώρησης από τον barrier.}, η κλιμακωσιμότητα των κεντρικοποιημένων κλήσεων φραγής σε μεγάλα συστήματα κάποιων εκατοντάδων επεξεργαστών δεν είναι ικανοποιητική \cite{mellor1991algorithms}.

Τις συνθήκες υψηλού ανταγωνισμού που υπάρχουν στις κεντρικοποιημένες κλήσεις φραγής μπόρεσε να αντιμετωπίσει ο \texttt{combining tree barrier}. Η ιδέα πίσω από αυτή την υλοποίηση είναι η ύπαρξη πολλών μικρών barriers που αποτελούν τους κόμβους-φύλλα μιας δεντρικής δομής δεδομένων και στους οποίους ανατίθενται γνήσια υποσύνολα νημάτων. Κάθε κόμβος τοποθετείται σε διαφορετικά τμήματα μνήμης (memory modules), έτσι ώστε οι προσβάσεις στη μνήμη να μοιράζονται μεταξύ των κόμβων και να μην εστιάζονται σε μία μόνο τοποθεσία όπως συμβαίνει με τους κεντρικοποιημένους barriers. Το τελευταίο νήμα που θα φτάσει σε κάθε barrier-φύλλο (leaf barrier) συνεχίζει διαδίδοντας ενημερώσεις προς τη ρίζα του δέντρου. Όταν κάποιο νήμα καταφέρει να φτάσει στη ρίζα του δέντρου, αυτό σημαίνει ότι όλα τα νήματα που συμμετέχουν στον παράλληλο υπολογισμό έχουν φτάσει στον barrier και η διαδικασία απελευθέρωσής τους από αυτόν μπορεί να ξεκινήσει. Κατά τη διαδικασία απελευθέρωσης, το νήμα που έφτασε στη ρίζα του δέντρου ξεκινά να διαδίδει ενημερώσεις προς τα φύλλα ώστε να επιτραπεί στα νήματα που περιμένουν να συνεχίσουν την εκτέλεση τους.

Ένας παρόμοιος αλγόριθμος με αυτόν του \texttt{combining tree barrier} στον οποίο χρησιμοποιείται δεντρική δομή που έχει τη μορφή δυαδικού δέντρου αποτελεί ο \texttt{tournament barrier}. Ο συγχρονισμός μεταξύ των νημάτων πραγματοποιείται μεταξύ δύο νημάτων τη φορά, ενώ το νήμα-εκπρόσωπος που συνεχίζει στο ανώτερο επίπεδο καθορίζεται στατικά και όχι βάσει της σειράς άφιξης, αποφεύγοντας με αυτό τον τρόπο τη χρήση ατομικών εντολών που υλοποιούνται στο υλικό (π.χ. \texttt{fetch\_and\_add()}) και χρησιμοποιούνται από τον \texttt{combining tree barrier}.

Οι παραλλαγές των αλγορίθμων κλήσεων φραγής είναι πάρα πολλές και διαφέρουν άλλες περισσότερο και άλλες λιγότερο μεταξύ τους, ανάλογα με τους στόχους που θέλουν να επιτύχουν αυτοί που τους επινόησαν. Για παράδειγμα, κάποιος μπορεί να προτιμά τη χρήση ή μη ατομικών εντολών που υλοποιούνται στο υλικό (\texttt{fetch\_and\_add()}, \texttt{compare\_and\_swap()} κλπ), τον στατικό ή μη ορισμό των διευθύνσεων μνήμης των κόμβων ενός \texttt{combining tree barrier}, τη χρήση ή μη κλειδαριών (locks) κόκ. Οι διάφορες παραλλαγές στοχεύουν στην επίλυση ζητημάτων που σχετίζονται για παράδειγμα με τις παράλληλες εφαρμογές ή την οργάνωση των υποκείμενων συστημάτων και μπορεί να αφορούν πλην άλλων το πλήθος τον απομακρυσμένων προσβάσεων μνήμης σε συστήματα NUMA, την πολυπλοκότητα χώρου που χρειάζεται ο barrier (π.χ. $O(1)$ για χρήση σταθερού πλήθους κοινόχρηστων μεταβλητών ή $O(N)$ όπου $N$ το πλήθος των νημάτων για χρήση πινάκων με κάθε στοιχείο να αντιστοιχεί σε ένα νήμα) ή το ποσό της κίνησης που δημιουργείται στο δίκτυο διασύνδεσης λόγω των πρωτοκόλλων συνοχής κρυφής μνήμης ανάλογα το είδος αυτών.

\section{Barriers στο OpenMP}
\label{sec:Barriers in OpenMP}
Στη διεπαφή προγραμματισμού εφαρμογών OpenMP που περιγράφηκε στο Κεφάλαιο \ref{ch:OpenMP API}, είδαμε ότι οι κλήσεις φραγής χρησιμοποιούνται ευρέως, τόσο έμμεσα (implicit barrier) στο τέλος παράλληλων περιοχών και των περιοχών διαμοιρασμού εργασίας, όσο και άμμεσα (explicit barrier) μέσω της οδηγίας \texttt{barrier}. Συνεπώς, το πόσο αποδοτική ή όχι είναι η υλοποίηση του barrier μπορεί να έχει σημαντικό αντίκτυπο στην συνολική επίδοση της παράλληλης εφαρμογής. % maybe move to next section

Ενδιαφέρον αποτελεί το γεγονός ότι ο ρόλος των κλήσεων φραγής στα πλαίσια του OpenMP είναι διευρυμένος σε σχέση με τις κλήσεις φραγής που περιγράψαμε μέχρι στιγμής. Αυτό συμβαίνει καθώς ο ρόλος τους εκτός από την επίτευξη συγχρονισμού μεταξύ των νημάτων, περιλαμβάνει και την εξασφάλιση ολοκλήρωσης των εργασιών OpenMP (Υποενότητα \ref{ssec:task directive}), των λεγόμενων και ως OpenMP tasks ή απλώς tasks. Συνεπώς, οι προϋπάρχοντες αλγόριθμοι κλήσεων φραγής που στοχεύουν μόνο στο συγχρονισμό των νημάτων, είτε θα πρέπει να προσαρμοστούν κατάλληλα για την εξασφάλιση της ολοκλήρωσης των tasks, είτε θα πρέπει να επαναδιατυπωθούν από την αρχή για λόγους καλύτερης σχεδίασης, που εν τέλει μπορεί να επηρεάσουν την απλότητα και την απόδοση της υλοποίησης.

\section{Ο barrier του OMPi}
\label{sec:OMPi's barrier}
Η διπλωματική εργασία περιέχει $\nu$ κεφάλαια.

\section{Βελτιώσεις που έγιναν στον barrier του OMPi}
\label{sec:Improvements in OMPi's barrier}
Η διπλωματική εργασία περιέχει $\nu$ κεφάλαια.
