\chapter{Συγχρονισμός με barriers}
\label{ch:Synchronization with Barriers}
Όπως έχουμε ήδη αναφέρει σε προηγούμενα κεφάλαια, ο προγραμματισμός συστημάτων κοινόχρηστου χώρου διευθύνσεων βασίζεται συνήθως στη χρήση νημάτων, ο συγχρονισμός μεταξύ των οποίων είναι ιδιαίτερα σημαντικός καθώς εξασφαλίζει τη συνέπεια των δεδομένων και την ορθότητα του προγράμματος.

Μία από τις πιο διαδεδομένες μεθόδους συγχρονισμού είναι η \textit{κλήση φραγής}, γνωστή και ως barrier. Όταν στον πηγαίο κώδικα υπάρχει μία κλήση φραγής και συναντηθεί από ένα νήμα, τότε το νήμα περιμένει σε αυτό το σημείο την άφιξη όλων των υπόλοιπων νημάτων στο ίδιο σημείο, πριν μπορέσουν όλα μαζί να συνεχίσουν την εκτέλεση του κώδικα που ακολουθεί την κλήση φραγής.

Οι κλήσεις φραγής εκτός από νήματα μπορούν να χρησιμοποιηθούν και για το συγχρονισμό διεργασιών, κάτι που όμως δεν θα μας απασχολήσει στο πλαίσιο της παρούσας διπλωματικής εργασίας.

\section{Υλοποιήσεις barrier}
\label{sec:Barrier Implementations}
Παρόλο που η λογική των κλήσεων φραγής είναι αρκετά απλή, υπάρχει πληθώρα αλγορίθμων για την υλοποίησή τους.

\section{Barriers στο OpenMP}
\label{sec:Barriers in OpenMP}
Στη διεπαφή προγραμματισμού εφαρμογών OpenMP που περιγράφηκε στο Κεφάλαιο \ref{ch:OpenMP API}, είδαμε ότι οι κλήσεις φραγής χρησιμοποιούνται ευρέως, τόσο έμμεσα (implicit barrier) στο τέλος παράλληλων περιοχών και των περιοχών διαμοιρασμού εργασίας, όσο και άμμεσα (explicit barrier) μέσω της οδηγίας \texttt{barrier}. Συνεπώς, το πόσο αποδοτική ή όχι είναι η υλοποίηση του barrier μπορεί να έχει σημαντικό αντίκτυπο στην συνολική επίδοση της παράλληλης εφαρμογής. % maybe move to next section

Ενδιαφέρον αποτελεί το γεγονός ότι ο ρόλος των κλήσεων φραγής στα πλαίσια του OpenMP είναι διευρυμένος σε σχέση με τις κλήσεις φραγής που περιγράψαμε μέχρι στιγμής. Αυτό συμβαίνει καθώς ο ρόλος τους εκτός από την επίτευξη συγχρονισμού μεταξύ των νημάτων, περιλαμβάνει και την εξασφάλιση ολοκλήρωσης των εργασιών OpenMP (Υποενότητα \ref{ssec:task directive}), των λεγόμενων και ως OpenMP tasks ή απλώς tasks. Συνεπώς, οι προϋπάρχοντες αλγόριθμοι κλήσεων φραγής που στοχεύουν μόνο στο συγχρονισμό των νημάτων, είτε θα πρέπει να προσαρμοστούν κατάλληλα για την εξασφάλιση της ολοκλήρωσης των tasks, είτε θα πρέπει να επαναδιατυπωθούν από την αρχή για λόγους καλύτερης σχεδίασης, που εν τέλει μπορεί να επηρεάσουν την απλότητα και την απόδοση της υλοποίησης.

\section{Ο barrier του OMPi}
\label{sec:OMPi's barrier}
Η διπλωματική εργασία περιέχει $\nu$ κεφάλαια.

\section{Βελτιώσεις που έγιναν στον barrier του OMPi}
\label{sec:Improvements in OMPi's barrier}
Η διπλωματική εργασία περιέχει $\nu$ κεφάλαια.
